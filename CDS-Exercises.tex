%%%%%%%%%%%%%%%%%%%%%%%%%%%%%%%%%%%%%%%%%%%%%%%
% CL provided template
%%%%%%%%%%%%%%%%%%%%%%%%%%%%%%%%%%%%%%%%%%%%%%%

\documentclass[12pt,a4paper,oneside,openright]{report}

% makes subsubsections appear in the toc
\setcounter{secnumdepth}{3}
\setcounter{tocdepth}{3}

% turns references into hyperlinks
\usepackage[pdfborder={0 0 0}, colorlinks=true, urlcolor=cyan]{hyperref}
\newcommand{\URL}[1]{\href{https://#1}{\textcolor{cyan}{\texttt{#1}}}}

\usepackage{tablefootnote}

% adjusts page layout
\usepackage[margin=25mm]{geometry}

% allows inclusion of PDF, PNG and JPG images
\usepackage{graphicx}
\graphicspath{ {figs/} }

\usepackage{verbatim}

% itemize with multiple columns
\usepackage{multicol}

% try to avoid widows and orphans
\raggedbottom
\sloppy
\clubpenalty1000%
\widowpenalty1000%

% adjust line spacing to make more readable
\renewcommand{\baselinestretch}{1.1}

% add colour to TODOs
\usepackage{xcolor}

\renewcommand{\chaptername}{Supervision}

%%%%%%%%%%%%%%%%%%%%%%%%%%%%%%%%%%%%%%%%%%%%%%%
% Maths
%%%%%%%%%%%%%%%%%%%%%%%%%%%%%%%%%%%%%%%%%%%%%%%

\usepackage{amsmath}        % American Mathematical Society
\usepackage{amssymb}        % Maths symbols
\usepackage{amsthm}         % Theorems
\usepackage{mathpartir}    % Proofs and inference rules

%%%%%%%%%%%%%%%%%%%%%%%%%%%%%%%%%%%%%%%%%%%%%%%
% Floats
%%%%%%%%%%%%%%%%%%%%%%%%%%%%%%%%%%%%%%%%%%%%%%%

\usepackage{float}
\usepackage[labelfont=bf,margin=10pt]{caption}
\usepackage{subcaption}
\newcommand{\mycaption}[2]{\caption[#1]{#1 #2}}

\floatstyle{plain}
\restylefloat{figure}
\restylefloat{table}

%%%%%%%%%%%%%%%%%%%%%%%%%%%%%%%%%%%%%%%%%%%%%%% 
% Figures
%%%%%%%%%%%%%%%%%%%%%%%%%%%%%%%%%%%%%%%%%%%%%%%

\usepackage{tikz}
\usepackage{pgfplots}

\newcommand{\graphWidth}{15cm}

% Shamelessly stolen to draw VBox diagrams
\usepackage{threads}

%%%%%%%%%%%%%%%%%%%%%%%%%%%%%%%%%%%%%%%%%%%%%%%
% Tables
%%%%%%%%%%%%%%%%%%%%%%%%%%%%%%%%%%%%%%%%%%%%%%%

\usepackage{array}
% control width and vertically align text in table cells
\newcolumntype{L}[1]{>{\raggedright\let\newline\\\arraybackslash\hspace{0pt}}p{#1}}
\newcolumntype{C}[1]{>{\centering\let\newline\\\arraybackslash\hspace{0pt}}p{#1}}
\newcolumntype{R}[1]{>{\raggedleft\let\newline\\\arraybackslash\hspace{0pt}}p{#1}}

%%%%%%%%%%%%%%%%%%%%%%%%%%%%%%%%%%%%%%%%%%%%%%%
% Listings
%%%%%%%%%%%%%%%%%%%%%%%%%%%%%%%%%%%%%%%%%%%%%%%

\usepackage{fancyvrb}

\floatstyle{boxed}
%\floatstyle{ruled}
\newfloat{Listing}{tbp}{lol}[chapter]

\DefineVerbatimEnvironment{JavaCode}{Verbatim}
{fontfamily=courier,baselinestretch=1,gobble=4}

\DefineVerbatimEnvironment{GoCode}{Verbatim}
{fontfamily=courier,baselinestretch=1,gobble=4}

\usepackage{listings}
\usepackage{color}
\usepackage{mathtools}

\definecolor{dkgreen}{rgb}{0,0.4,0}
\definecolor{gray}{rgb}{0.5,0.5,0.5}
\definecolor{mauve}{rgb}{0.58,0,0.82}

\lstset{
  language=Java,
  aboveskip=0mm,
  belowskip=0mm,
  showstringspaces=false,
  %columns=flexible,
  basicstyle={\small\ttfamily},
  numbers=left,
  keywordstyle=\bfseries,
  commentstyle=\color{dkgreen},
  stringstyle=\color{mauve},
  breaklines=true,
  breakatwhitespace=true,
  tabsize=4
}

%%%%%%%%%%%%%%%%%%%%%%%%%%%%%%%%%%%%%%%%%%%%%%%
% Question formatting
%%%%%%%%%%%%%%%%%%%%%%%%%%%%%%%%%%%%%%%%%%%%%%%

\newcommand{\question}[2]{\paragraph{#1} #2}

%%%%%%%%%%%%%%%%%%%%%%%%%%%%%%%%%%%%%%%%%%%%%%%
% Semantic and convenience macros
%%%%%%%%%%%%%%%%%%%%%%%%%%%%%%%%%%%%%%%%%%%%%%%

\newcommand{\todo}[1]{\textcolor{red}{TODO: #1}}
\newcommand{\note}[1]{\textcolor{blue}{NOTE: #1}}

\newcommand{\javaLiteral}[1]{\texttt{#1}}
\newcommand{\javaCode}[1]{\texttt{#1}}
\newcommand{\javaClass}[1]{\texttt{#1}}
\newcommand{\javaSlot}[1]{\texttt{#1}}
\newcommand{\javaVariable}[1]{\texttt{#1}}
\newcommand{\javaKeyword}[1]{\texttt{#1}}
\newcommand{\javaMethod}[1]{\texttt{#1}}
\newcommand{\javaException}[1]{\texttt{#1}}
\newcommand{\javaAnnotation}[1]{\texttt{#1}}

\newcommand{\keyTerm}[1]{\textbf{#1}}

\newcommand{\codeemph}[1]{\textbf{#1}}

%%%%%%%%%%%%%%%%%%%%%%%%%%%%%%%%%%%%%%%%%%%%%%%
% Info about document
%%%%%%%%%%%%%%%%%%%%%%%%%%%%%%%%%%%%%%%%%%%%%%%

\newcommand{\course}{Concurrent and Distributed Systems}

\title{Concurrent and Distributed Systems\\ Exercise Sheet}
\date{\today}
\author{Mike Cachopo}

%%%%%%%%%%%%%%%%%%%%%%%%%%%%%%%%%%%%%%%%%%%%%%%
% Start of document
%%%%%%%%%%%%%%%%%%%%%%%%%%%%%%%%%%%%%%%%%%%%%%%

\begin{document}

%%%%%%%%%%%%%%%%%%%%%%%%%%%%%%%%%%%%%%%%%%%%%%%
% Cover page
%%%%%%%%%%%%%%%%%%%%%%%%%%%%%%%%%%%%%%%%%%%%%%%

\pagestyle{empty}

\maketitle

\newpage

\pagestyle{plain} \pagenumbering{roman}

\section*{Submitting work}

Submit your answers to the questions as a single PDF file by email. I
am rarely in Cambridge so I can't accept paper submissions. I strongly
suggest using LaTeX to typeset your answers --- you'll need it next
year for your dissertation, and the sooner you start practicing the
better. I recommend the Ti\textit{k}Z package to make diagrams,
but hand-drawn and scanned is fine if you are in a rush --- you should
prioritise your learning over making your supervision work look
pretty.

The sooner you submit work the more time I have to mark it and give
you detailed feedback. Aim to submit your work 48 hours before the
corresponding supervision. If you can't submit work in advance, please
let me know not to wait for it. I am happy to mark work submitted
after a supervision, or to mark a new attempt at the work after the
supervision if you want to fix your mistakes. Keep in mind that you'll
be missing out on most of the benefit of the supervision if you don't
even attempt the work beforehand.

\section*{Questions}

Each supervision is split into three sections:

\subsection*{Bookwork}

\textbf{Do not include the answers to these questions in your work!}

These are questions that you should be able to answer after going
through the lectures. We won't have time to cover them all, but I will
ask you some of them during the supervisions. I strongly recommend the
questions that involve reading textbooks --- reading alternative
explanations of the content will give you a deeper understanding.

Use these questions to make sure you understand the key concepts of
the course. If there is any question that you don't understand, please
raise it at the start of the supervision, so we can discuss it.

When you are asked these questions during supervisions, use them to
give you practice explaining computer systems in a precise and concise
way. This is a skill you will need to answer Tripos questions on a
variety of courses, if you ever write academic papers on systems
research, or if you ever have to discuss system design with colleagues
in an engineering role.
  
\subsection*{Exercises}

These are mostly Tripos questions. Please submit written answers to
these according to the instructions above. Please answer them as if
you were writing your exam, except for the time limit.

Examiners have to mark 100+ scripts per course in just a few
days. They can only spend a few minutes reading your answer,
so you should make it as obvious as possible that you understand the
material. This means that your answer should be correct, precise, and
well structured. Writing down key words won't get you very far unless
you can clearly explain how they help solve the problem.

%% \begin{itemize}
%% \item Start by answering each part of the question with a few key
%%   points. This is a good way to pick up 5-10 marks in only a couple of
%%   minutes.
%% \item Go back and develop well-reasoned arguments where the question
%%   asks you to compare two possible solutions to a problem, or add
%%   relevant details to a description of a particular system's
%%   implementation. This should put you in the 10-15 mark range with
%%   about 10 extra minutes of work.
%% \item Finally, add detail where you think it is relevant or may show
%%   deeper understanding. This is a great opportunity to show off some
%%   \textbf{relevant} further reading you've done, and should put you
%%   comfortably in the 15-20 mark range.
%% \end{itemize}

%% This process takes practice, so expect to need a few attempts before
%% you get comfortable with it. In an exam, you may well choose to settle
%% for the 10-15 mark range, or even 5-10, so that you can spend more
%% time on another question. This kind of game of optimisation may seem
%% sub-optimal to maximise your learning, but sadly Tripos exams are not
%% optimal learning tools either, and keeping these strategies in mind
%% will help you maximise your grades.
  
\subsection*{Extensions}

The extension sections contain further reading material and questions,
which you are welcome to read if you have spare time and/or are
particularly interested in the course. This material is not
examinable, but understanding it will give you more perspective on the
contents of the course. It will prepare you for more advanced
systems courses, units of assessment, and/or a career involving
designing and building concurrent and distributed systems.

I do not expect you to write answers to any questions in this section,
but I am happy to discuss them in supervision after we cover the
examinable material. If there are other advanced topics you would like
to discuss, please let me know when you submit your work so that I can
prepare in advance.

\section*{General advice}

As an undergraduate in Cambridge, I often felt overwhelmed by all the
work to be done. If there is one piece of advice I can pass down to
you, is to take care of your physical and mental health first, and
only then worry about your work.

If you're struggling, please do not hesitate to reach out to me, your
DoS and your tutor. It is our job to make sure you get the most out of
your education, and we can often help you if you ask for help.

Lastly, enjoy learning through these questions!

\newpage

% \tableofcontents

%%%%%%%%%%%%%%%%%%%%%%%%%%%%%%%%%%%%%%%%%%%%%%%
% #Content
%%%%%%%%%%%%%%%%%%%%%%%%%%%%%%%%%%%%%%%%%%%%%%%


\chapter{Concurrency primitives}
\pagestyle{headings} \pagenumbering{arabic}

\section{Bookwork}

\question{The free lunch is over}{Before trying to solve all the
  problems of concurrent programming, it's important to understand why
  concurrency is desirable in the first place. Read
  \cite{FreeLunchIsOver}. We will discuss in the supervision.}

\question{Lecture 1 concepts}{Define and explain the following:
  \begin{itemize}
  \item Multicore processors;
  \item Processes, threads and the differences between them (you may
    want to mention the memory stacks and heap), as well as how
    context switches work;
  \item Concurrency, and the difference between time sharing a single
    core and true parallelism;
  \item User-level, kernel-level and hybrid thread scheduling, the
    advantages and disadvantages of each as well as your favourite
    language that uses each of them;
  \item ``Mostly correct'' code and why it is a problem;
  \item Critical sections and mutual exclusion;
  \item Race conditions and atomic memory operations.
  \end{itemize}
}

\question{Lecture 2 concepts}{
  \begin{itemize}
  \item In a multicore system, would it be possible to achieve mutual
    exclusion on critical sections by locking (having exclusive access
    to) the memory bus when entering a critical section? You may find
    it useful to refer to the slide titled ``Computer Architecture
    Reference Models''.
  \item What can cause a running process/thread to block?
  \item What are the problems with disabling interrupts to provide
    atomic execution of a critical section? Don't forget the ones that
    do not involve other cores or DMA.
  \item Why do we need special ISA instructions to make an atomic
    read-and-set? Are these instructions strictly necessary to write
    correct concurrent programs? You may want to mention Lamport's
    bakery algorithm in your answer.
  \item How would you explain CAS to someone who just finished Part IA
    of the Computer Science Tripos?
  \item How would you explain LL/SC to someone who just finished Part
    IA of the Computer Science Tripos?
  \item What is the difference between deadlock and livelock?
  \item Be prepared to act out the dining philosophers' problem in the
    supervision and explain where the problem lies. Can you think of a
    simple solution?
  \end{itemize}
}

\question{Lecture 3 concepts}{
  \begin{itemize}
  \item Why is sleeping preferable to spinning?
  \item Describe the implementation of a semaphore and its
    operations. Refer to
    \cite[Chapters~10.4--10.6]{bacon2003operating} for more in-depth
    information than the lecture slides.
  \item What is an IPI? For this course you don't need to know much
    about low-level implementation details, but you should be
    comfortable explaining what they are and how they can be used at a
    high level.
  \item How can a semaphore be used to provide mutual exclusion,
    condition synchronisation and resource allocation? What is the
    starting value of the semaphore and what are the logical meanings
    of the operations in each case?
  \item Explain the producer-consumer problem. What are some
    real-world applications of this problem? How can you solve it with
    semaphores? How would you solve it with locks? You may wish to
    refer to \cite[Chapter~11.3]{bacon2003operating}.
  \end{itemize}
}

\question{Lecture 4 concepts}{
  \begin{itemize}
  \item It is safe for multiple threads to concurrently read a shared
    resource, but writes must have exclusive access. Why?  What can go
    wrong if a read and a write happen concurrently?  What can happen
    if two writes happen concurrently?
  \item Be ready to sketch out and discuss several RW lock
    implementations using semaphores in the supervision, focusing on
    their starvation and fairness properties. This should include the
    implementation in \cite[Chapter~11.5]{bacon2003operating}.
  \item What are the advantages of Conditional Critical Regions, and
    why do we care about them? Why is CCR not a common synchronisation
    mechanism?
  \item Where are monitors used in Java, and how do they work? This is
    covered in the \emph{Further Java} companion course. If you
    haven't gotten there yet have a look at the Java specification.
  \item Why are condition variables a useful complement to monitors?
    What are the similarities and differences with semaphores? How are
    they used in Java?
  \item What are the pros and cons of signal-and-wait and
    signal-and-continue semantics? Which is used in Java? What
    implications does this have on how we \texttt{wait} in Java?
  \item How would you implement a RW lock using a monitor? (Hint:
    adapt the implementation using semaphores in
    \cite[Chapter~11.5]{bacon2003operating}.)
  \item What concurrency primitives do C++ pthreads support?
  \end{itemize}
}


\section{Exercises}

\question{Reading and
  writing}{\href{https://www.cl.cam.ac.uk/teaching/exams/pastpapers/y2010p5q4.pdf}{2010
    Paper 5 Question 4}}

\question{Semaphores and
  monitors}{\href{https://www.cl.cam.ac.uk/teaching/exams/pastpapers/y2000p3q1.pdf}{2000
    Paper 3 Question 1}}

\section{Extensions}

\question{Real-life synchronisation}{Explain to me the solution to
  \cite[Chapter~1,~Exercise~4]{ArtMultiprocessorProgramming}.}

\question{Java concurrency}{Ask me any question about concurrent
  programming in Java that I cannot answer. (You will most likely
  succeed if you try hard enough, but I'm curious to see what you'll
  come up with.) You may find it useful to refer to the
  \href{https://docs.oracle.com/javase/specs/jls/se15/html/jls-17.html}{Java
    specification}.}

\question{Double-checked locking}{Read \cite{DCL}. What is your
  favourite solution to the double-checked locking problem with the
  singleton pattern in Java?}

\question{Progress guarantees}{What does it mean for an application to
  be:
  \begin{itemize}
  \item lock-free;
  \item wait-free;
  \item obstruction-free?
  \end{itemize}
  What does it mean for a particular operation within an application
  to have these properties?  }

\chapter{Transactions}

\section{Bookwork}

\question{Lecture 5 concepts}{
  \begin{itemize}
  \item This is repeated from last supervision, but it's worth making
    sure you remember: What is the difference between deadlock and
    livelock?
  \item What are the requirements for the possibility of deadlock? How
    is each of them shown in the ``Resource allocation graph'' in the
    slides? How would you remove each of the requirements?
  \item If you were writing a concurrent application, how would you
    deal with deadlock? This question is deliberately very open-ended,
    we will discuss in supervision.
  \item One way to prevent hold-and-wait is to request all resources
    simultaneously. Why can this be a hard thing to do?
  \item Again, this is a repeat from last supervision, but now you
    should make sure you can do it: What is a simple solution to the
    dining philosophers problem?
  \item \todo{Banker's algorithm?}
  \item What are some pros and cons of the thread scheduling policies
    on the slides?
  \item Sketch out an example scenario of priority inversion without
    using a reference. How would priority inheritance help here?
  \end{itemize}
}

\question{Lecture 6 concepts}{
  \begin{itemize}
  \item Why are higher-level concurrent programming models useful?
    What higher-level models are there, and what are some examples
    (languages that use them, libraries, applications, etc) of each?
  \item Explain the programming model of active objects. What is it
    similar to? How does it provide mutual exclusion and condition
    synchronisation?
  \item Explain message passing. How does it differ from active
    objects? What is the difference between synchronous and
    asynchronous message passing?
  \item Why are ``composite operations'' important? What are some
    examples of applications that may use them?
  \item Explain the problems shown in the composite operation example
    in the slides.
  \item What are the ACID properties?
  \item What is serialisability? How is it different from a serial
    execution?
  \end{itemize}
}

\question{Lecture 7 concepts}{
  \begin{itemize}
  \item What kinds of operation conflicts are there? What are the
    problems caused by each?
  \item What is the difference between strict isolation and non-strict
    isolation?
  \item What are cascading aborts, and why are they a problem?
  \item How does two-phase locking work? What algorithmic difference
    is there between a version of 2PL that offers non-strict isolation
    and one that offers strict isolation? Can you think of a way to
    avoid the need to know the locks required in advance?
  \item Why might a transaction abort?
  \item What approaches are there to rollback a transaction?
  \item How does timestamp ordering work? What timestamp is given to a
    transaction as it begins? At commit time, what needs to be
    checked? What are its pros and cons compared with 2PL?
  \item What is optimistic concurrency control? You may want to read
    \cite[Chapter~20.6]{bacon2003operating} for this one, because the
    slides are somewhat misleading. What are its advantages and
    disadvantages?
  \item Explain the OCC algorithm used in the slides. What's wrong
    with the ``OCC example (2)'' slide? As in, how is it being overly
    conservative? (Hint: can a write-only transaction conflict with
    any committed transaction?)
  \item In an application with a high degree of contention --- many
    concurrent transactions conflict, forcing each other to abort ---
    what is the best synchronisation mechanism?
  \end{itemize}
}

\question{Lecture 8 concepts}{
  \begin{itemize}
  \item What is a crash? What can cause it? What assumptions does the
    fail-stop model make, and how can they be broken?
  \item Why does ``writing all objects to disk on commit'' not work?
    How is write-ahead logging different?
  \item What is the bare minimum information that entries in a
    write-ahead log must hold? What does that information let us do
    when a crash occurs?
  \item Why is lazy write back desirable? What problems does it
    introduce, and how are they solved with a well placed flush? Does
    batching change the guarantees provided by the log?
  \item What benefits do checkpoints give us? In supervision, draw a
    diagram and explain the algorithm for restarting from a
    checkpoint.
  \item In 60 seconds or less, sell me a transactional system. As in,
    convince me that transactions are a ``good'' model to use to write
    a concurrent application (you have to tell me what ``good'' means
    too!)
  \item What are the pros and cons of lock-free programming? What is
    linearisability? If I asked you to implement a lock-free data
    structure, such as a linked list, what would you reply?
  \item What is transactional memory? Why is it ``promising''? Why is
    it not widely used today?
  \end{itemize}
}


\section{Exercises}

\question{TSO and
  OCC}{\href{https://www.cl.cam.ac.uk/teaching/exams/pastpapers/y2007p4q1.pdf}{2007
    Paper 4 Question 1}}

\question{TSO
  details}{\href{https://www.cl.cam.ac.uk/teaching/exams/pastpapers/y2011p5q7.pdf}{2011
    Paper 5 Question 7}}

\section{Extensions}

\question{Message passing}{If you want to know more about message
  passing models read up on Hoare's Communicating Sequential Processes
  (CSP) or Milner's Communicating Sequential Systems (CCS). If you
  would like a reasonably clean introduction to how CSP can be used in
  a programming language, have a look at
  \href{https://tour.golang.org}{A Tour of Go}, in particular the
  section on channels.}

\question{BASE}{What are the BASE properties? Why would they be more
  desirable than ACID?}

\question{Lock-free programming}{The \emph{Multicore Semantics and
    Programming} Part II Unit of Assessment has a two hour lecture on
  lock-free programming. If you want to know more, it's a good place
  to start.}

\question{Transactional memory}{For a relatively short intro to
  Software Transactional Memory, check out
  \cite[Chapter~18]{ArtMultiprocessorProgramming}. For a deeper
  understanding of why STM is a very powerful tool, read \cite{CMT}.}


\chapter{Distributed systems and ordering}

\section{Bookwork}

\question{Introduction to distributed systems}{Let's talk about
  distributed systems from an informal perspective. Some topics to
  cover:
  \begin{itemize}
  \item Intuitively, what is a distributed system? Why are some
    applications distributed? What are some examples?
    % Refer to slide 6
  \item You will learn a lot about TCP/IP, DNS, HTTP and network
    latency and throughput in the Networks course in Lent term. For
    now, let's just make sure you know the basics needed to understand
    communication between processes.
    % Reliable in-order delivery, naming are the most important here
  \item RPC is a very powerful and widely used framework. It is not
    very relevant at the level of abstraction usually kept in this
    course, but you will likely encounter it in many software
    engineering roles. Let's discuss this briefly. Do you see the
    similarity with active objects, from the concurrent part of the
    course? What are the key differences?
    % Talk about marshalling and Java's serialisable, proto buffers,
    % JSON, and compatibility between languages
  \item Distributed systems are extremely complex, and there are a lot
    of problems to solve in the area. Let's briefly talk about the
    specific problems addressed in this course.
    % Ordering of events (using clocks), ordered broadcast over
    % point-to-point links, consensus, replication for fault
    % tolerance, availability and consistency (briefly 2-phase commit)
  \end{itemize}
}

\question{Models of distributed systems}{Now that you have an
  intuitive feel about distributed systems, let's discuss some ways in
  which we can model them. Some topics to cover:
  \begin{itemize}
  \item How would you describe the two generals problem to your
    Engling friend? This is an application of what general problem in
    distributed systems? Can you think of another application?
    % Both students should attempt this; consensus; the example in the
    % slides is acceptable
  \item What assumption made in the two generals problem is no longer
    true in the Byzantine generals problem? Can you think of a
    real-world application of this problem?
    % Blockchain: processes have an incentive to lie, to make more
    % money; mention crypto as a possible solution
  \item What are the three types of behaviour we have to model in a
    distributed system? What are the most common models of each type,
    and what are their properties? Which can we assume in the
    real-world?
    % Network: reliable, fair-loss, arbitrary/adversary links
    % Node: crash-stop, crash-recovery, byzantine
    % Timing: synchronous, partially synchronous, asynchronous
  \item We will talk more about availability later, when we talk about
    consistency, but feel free to ask any questions you have now.
  \item What are faults and failures? Why is the difference important?
    How do we detect them in practice?
    % Fault is single node/link, failure is system wide; timeouts
  \end{itemize}
}

\question{Time, clocks and ordering}{In the concurrency part of the
  course, we saw that being able to order events is quite a neat
  ability. For example, when we attempt to serialise transactions,
  it's useful to be able to say that one transaction is serialised
  \emph{before} another. As it turns out, ordering is much more
  difficult in a distributed system.
  \begin{itemize}
  \item Read the first two pages of Lamport's paper on time
    \cite{lamport1978time}. Why must orderings in a distributed system
    be only partial orders?
  \item What kinds of physical clocks are there? What is each used
    for? What are the shortcomings of each?
    % Quartz clocks (50ppm), atomic (1E-14), mention GPS
  \item What representations of physical time are there? How is each
    of them useful?
    % UTC is International atomic time (TAI) + corrections for Earth's
    % rotation (leap seconds); Unix time (seconds since epoch); ISO
    % 8601 date+time
  \item What is the purpose of NTP? How does it achieve this goal? In
    words, how does an NTP client estimate clock skew from an NTP
    server? What assumptions are made here? Are they fair assumptions
    to make?
    % Clock synchronisation; Stratum 0 is clock, 1 is server with
    % clock; slew if <125ms, else step if <1000s, else panic
  \item Why might it be useful to order messages in a distributed
    system? Can you think of a real-world example that does not
    involve discourse about cheese?
    % Distributed transaction system; node receives "update object X"
    % before it receives "create object X" from another node.
  \item How is a happens-before relationship on the events in a
    distributed system defined?
    % Local execution, message delivery, transitive closure
  \end{itemize}
}

\question{Broadcast protocols and logical time}{Now that we've seen
  how impractical physical time is in a distributed system, let's
  focus on logical time. A word of advice: whenever you start thinking
  about a logical clock, make sure you know what the clock is
  ordering. Is it ordering messages? Events? Transactions? Forgetting
  to state this explicitly will lead to misunderstandings and bugs
  that are difficult to pin down!
  % Mention this!
  \begin{itemize}
  \item In 60 seconds, explain Lamport clocks. You may want to
    mention:
    \begin{itemize}
    \item What is being ordered?
      % Local events
    \item What are the steps in the algorithm?
      % Increment on event; on receive max then increment
      % Mention similarity with the rules for building a hb relation
    \item How do Lamport clocks relate to a causal happens-before
      relationship?
      % a -> b implies L(a) < L(b), but not the opposite
    \item What does it mean for two events to have the same timestamp?
      % Either the same event, or completely unrelated at different
      % nodes
    \item How can a total order be formed from Lamport timestamps?
      % Make timestamps out of (L timestamp, node id)
    \end{itemize}
  \item As above, but now for vector clocks.
  \item In 30 seconds, explain the isomorphism between vector clocks
    and sets of events.
  \item What are the trade-offs between Lamport clocks and vector
    clocks? How does each of them scale with the number of processes
    and the amount of communication? How does each of them handle
    dynamic process groups?
    % Mention size, expressiveness, different types of orders
  \item What are broadcast protocols, and why are they useful?
  \item What is the difference between ``receiving'' and
    ``delivering'' a message? At which layer does each of them happen?
    Why is this distinction useful?
  \item Let's have a free-form discussion about ordering of broadcast
    messages. Please come prepared to explain each of the ordering
    schemes presented in the course, why they might be useful, and how
    they can be implemented.
    % FIFO: reliable link (retransmit + dedup) + pairwise seq number
    % Causal: essentially vector clocks
    % Total order: leader dictates order OR Lamport timestamps but
    % need to wait for a timestamp > T from each node before
    % delivering message with timestamp T
    % FIFO total order: FIFO + total order
    %
    % For causal ordering, mention that the algorithm is slightly
    % different depending on whether the timestamp entry of the
    % sending node is incremented before or after a message is
    % broadcast.
  \item Why is FIFO-total order broadcast strictly stronger than
    causal broadcast?
  \end{itemize}
}

\section{Exercises}

\question{Distributed
  time}{\href{https://www.cl.cam.ac.uk/teaching/exams/pastpapers/y2001p9q1.pdf}{2001
    Paper 9 Question 1}}

\question{Vector
  clocks}{\href{https://www.cl.cam.ac.uk/teaching/exams/pastpapers/y2005p9q4.pdf}{2005
    Paper 9 Question 4}}


\section{Extensions}

\question{GRPC}{If you end up writing parts of a distributed system,
  there's a good chance that you'll have to use GRPC or a similar
  framework. Feel free to do some research on it and discuss in
  supervision.}

\question{Catchup}{In the last supervision we will be covering
  consensus and distributed consistency, which are both very complex
  problems. Now is a good time for you to ask any questions you have
  on the course so far, as we are less likely to have time for them in
  the final supervision.}


\chapter{Topics in distributed systems}

\section{Bookwork}

In the last four lectures there are a lot of relatively isolated
topics, and we won't have time to cover them all in the
supervision. Instead of trying to go through all of them sequentially,
we will:
\begin{itemize}
\item Briefly discuss the motivations of replication;
\item You will have time to ask any questions you have about the
  course;
\item You will each choose two topics from the last four lectures and
  explain them in depth in the supervision, in the style of a short
  presentation.
\end{itemize}

Instructions for the topic presentations:
\begin{itemize}
\item Agree with your partner beforehand which topics each of you is
  presenting, so there is no overlap.
\item Aim to talk for about 5 minutes per topic, and allow for 1-2
  minutes of questions/discussion.
\item Prepare what you're going to say in advance. You may find it
  useful to have some written notes in front of you in the
  supervision.
\item Do some reading outside of the course for each of your
  topics. One paper or textbook chapter per topic is sufficient. The
  recommended textbooks and references section of the lecture notes
  are a good place to start.
\item You may want to choose one topic you're comfortable with, to
  focus on your presentation/technical discussion skills. These often
  transfer into the ability to write clear Tripos answers.
\item You may want to choose one topic you are struggling with, to
  give you the chance to become more familiar with it when preparing
  your presentation.
\end{itemize}


\question{Replication}{
  \begin{itemize}
  \item What is data replication? Why do we want it?
    % Data replication for fault tolerance
    % Mention RAID and sharding as context for different ways to
    % replicate data
    % Mention probability of single fault vs total failure
    % Mention computation replication
  \item Explain the complexity introduced by retrying state updates in
    a distributed system. You should cover:
    \begin{itemize}
    \item Why do retries exist in the first place?
      % Message loss, which is indistinguishable from ACK loss.
    \item Given that retries are necessary, what types of retry
      behaviour (semantics) can an application implement?\\
      You should state these as \emph{If a client sends a request,
        then the server will \textbf{deliver} that request $X$ times.}
      % At-most-once, at-least-once, exactly-once.
    \item What is idempotence? Why is it useful?
      % Makes it ok to deliver the same message several times in a
      % row.
    \item Can idempotence be implemented using message deduplication?
      This is an open-ended question; be ready to defend your
      argument.
      % Yes: dedup between receive and deliver means idempotence at
      % the receive level
      % No: dedup means messages are only delivered once, so
      % idempotence doesn't matter
    \end{itemize}
  \item Are exactly-once semantics ``enough'' in the presence of
    multiple clients or servers? Why?
    % Talk about different ways that operations can be reordered and
    % interleaved. Too long to discuss in depth. Ask if they
    % understand how timestamps and tombstones are used,
    % reconciliation, concurrent writes.
  \end{itemize}
}

% Ensure that the Kidz understand state machine replication using
% total order broadcast.

\question{Presentation topics}{As described above, choose two of the
  following:
  \begin{description}
  \item[Quorums] In particular, explain:
    \begin{itemize}
    \item The requirements of a quorum system, and why they are
      important;
    \item The performance tradeoffs of different choices of read and
      write quorum size;
    \item Different methods of achieving consistency, such as read
      repair.
      % Anti-entropy algorithms that periodically share latest
      % versions
    \end{itemize}
  \item[Consensus in general] I do not want you to explain Raft here,
    you're already doing that in one of the written exercises!
    Instead, I want you to focus on some of the following:
    \begin{itemize}
    \item How modern implementations of consensus algorithms differ
      from the traditional formulation of the problem;
      % Single value is easy to formalise
      % Multi-value is useful for FIFO-total order broadcast; they are
      % formally equivalent
    \item The PLF result;
    \item The motivations behind the creation of Raft, and how it
      differs from Paxos.
    \end{itemize}
  \item[Two-phase commit] Including:
    \begin{itemize}
    \item A comparison with consensus;
    \item A discussion of fault tolerance.
      % Algorithm on slides.
    \end{itemize}
  \item[Linearisability] Including how it relates to happens-before
    and to physical time.
  \item[The CAP theorem] In particular, explain:
    \begin{itemize}
    \item What each of the properties is, and how the meaning of
      consistency differs between contexts;
    \item Different possible consistency guarantees;
    \item The real-world implications of the CAP theorem;
    \item At least one of BASE and PACELC.
    \end{itemize}
  \item[Others] Any of the following, with the caveat that they are
    likely not examinable:
    \begin{itemize}
    \item CRDTs;
    \item A case study of any distributed system, like the one of
      Spanner in the slides. You may select Spanner or any other
      system that is relevant to the course.
    \end{itemize}
  \end{description}
}

\section{Exercises}

\question{Third topic}{Select one of the presentation topics from the
  bookwork section and explain it. This topic must not be one of the
  two you picked for the verbal presentation in supervision, but it
  can overlap with your supervision partner's.
  \\
  Do this exercise as if you were answering a 10 mark Tripos question
  part. The purpose of this exercise is for you to practice
  structuring medium-length Tripos answers, and in particular to
  select the most relevant information to include in your answer.  }

\question{Raft}{Explain in detail the purpose and operation of the
  Raft algorithm.
  \\
  Do this exercise as if you were answering a 20 mark Tripos
  question. The purpose of this exercise is for you to practice
  structuring long Tripos answers, writing clearly and concisely, and
  ensuring that even long answers are understandable.  You are not
  expected to remember all the details of Raft in a Tripos exam, so
  you should answer this question open-book.
  \\
  You may also find the
  \href{http://thesecretlivesofdata.com/raft/}{visualisation} of Raft
  useful as a source of inspiration.  }

% \question{Consistency}{\href{https://www.cl.cam.ac.uk/teaching/exams/pastpapers/y2004p8q4.pdf}{2004
% Paper 8 Question 4}}

% \question{Process
% naming}{\href{https://www.cl.cam.ac.uk/teaching/exams/pastpapers/y1998p12q1.pdf}{1998
% Paper 12 Question 1}}

% \question{RPC and
% ORB}{\href{https://www.cl.cam.ac.uk/teaching/exams/pastpapers/y1999p9q1.pdf}{1999
% Paper 9 Question 1}}

% \question{Access
% control}{\href{https://www.cl.cam.ac.uk/teaching/exams/pastpapers/y2000p8q1.pdf}{2000
% Paper 8 Question 1}}

\section{Extensions}

As mentioned in the bookwork section, the last four lectures of the
course cover a variety of topics, but each of them only briefly. If
you are interested, you can easily learn more about each of them, and
incorporate more of your findings in your topic presentations!

\bibliographystyle{apalike}

\bibliography{biblio}

\end{document}

%%% Local Variables:
%%% mode: latex
%%% TeX-master: t
%%% End:
